\documentclass{article} % For LaTeX2e
\usepackage{nips14submit_e,times}
\usepackage{amsmath}
\usepackage{amsthm}
\usepackage{amssymb}
\usepackage{mathtools}
\usepackage{hyperref}
\usepackage{url}
\usepackage{algorithm}
\usepackage[noend]{algpseudocode}
%\documentstyle[nips14submit_09,times,art10]{article} % For LaTeX 2.09

\usepackage{graphicx}
\usepackage{caption}
\usepackage{subcaption}

\def\eQb#1\eQe{\begin{eqnarray*}#1\end{eqnarray*}}
\def\eQnb#1\eQne{\begin{eqnarray}#1\end{eqnarray}}
\providecommand{\e}[1]{\ensuremath{\times 10^{#1}}}
\providecommand{\pb}[0]{\pagebreak}
\DeclarePairedDelimiter\ceil{\lceil}{\rceil}
\DeclarePairedDelimiter\floor{\lfloor}{\rfloor}

\newcommand{\E}{\mathrm{E}}
\newcommand{\Var}{\mathrm{Var}}
\newcommand{\Cov}{\mathrm{Cov}}

\def\Qb#1\Qe{\begin{question}#1\end{question}}
\def\Sb#1\Se{\begin{solution}#1\end{solution}}

\newenvironment{claim}[1]{\par\noindent\underline{Claim:}\space#1}{}
\newtheoremstyle{quest}{\topsep}{\topsep}{}{}{\bfseries}{}{ }{\thmname{#1}\thmnote{ #3}.}
\theoremstyle{quest}
\newtheorem*{definition}{Definition}
\newtheorem*{theorem}{Theorem}
\newtheorem*{lemma}{Lemma}
\newtheorem*{question}{Question}
\newtheorem*{preposition}{Preposition}
\newtheorem*{exercise}{Exercise}
\newtheorem*{challengeproblem}{Challenge Problem}
\newtheorem*{solution}{Solution}
\newtheorem*{remark}{Remark}
\usepackage{verbatimbox}
\usepackage{listings}
\usepackage{mathrsfs}
\title{Functional Analysis: \\
Problem Set I}


\author{
Youngduck Choi \\
CIMS \\
New York University\\
\texttt{yc1104@nyu.edu} \\
}


% The \author macro works with any number of authors. There are two commands
% used to separate the names and addresses of multiple authors: \And and \AND.
%
% Using \And between authors leaves it to \LaTeX{} to determine where to break
% the lines. Using \AND forces a linebreak at that point. So, if \LaTeX{}
% puts 3 of 4 authors names on the first line, and the last on the second
% line, try using \AND instead of \And before the third author name.

\newcommand{\fix}{\marginpar{FIX}}
\newcommand{\new}{\marginpar{NEW}}

\nipsfinalcopy % Uncomment for camera-ready version

\begin{document}


\maketitle

\begin{abstract}
This work contains solutions to the exercises of the problem set I.
\end{abstract}

\bigskip

\begin{question}[1]
\hfill
\begin{figure}[h!]
  \centering
    \includegraphics[width=0.7\textwidth]{funcA-h-e1-p1.png}
\end{figure}
\end{question}
\begin{solution} \hfill \\
First, define a map $\Phi:(E \setminus Y)^* \to Y^{\perp}$ naturally by
\eQb
f &\mapsto& (x \mapsto f([x]))
\eQe 
where $[x] \in E\setminus Y$. The map on the RHS is clearly linear and 
vanishes on $Y$, because for any $y \in Y$, $f([y]) = 0$
by linearity. The map is bounded as well, since
\eQb
|f([x])| &\leq& ||f||||[x]|| \leq ||f||||x|| \>\> (*)
\eQe 
for any $x \in E$.
Now, we show that $\Phi$ is an isometry.  
It follows that 
\eQb
||\Phi(f)|| &=& \sup_{||x|| = 1} |<\Phi(f),x>| = \sup_{||x|| = 1} |f([x])| \\
\eQe
Hence, combined with $(*)$, it suffices to show that
\eQb
\sup_{||x|| = 1} |f([x])| &\leq& \sup_{||[x]|| = 1} |f([x])|,
\eQe 
but this follows, since $Y$ is closed, so we can choose $\{y_n\} \subset Y$
such that $\limsup_{n \to \infty}||x + y_n|| \leq 1$.
Now, we show that $\Phi$ is surjective. Let $l \in Y^{\perp}$. Then, define
$f:E \setminus Y \to \mathbb{R}$ by
\eQb
[x] &\mapsto& l(x).
\eQe 
The map is well-defined, since for any $x' \in [x]$ such that $x' = x+y$,
$l([x']) = l(x + y) = l(x)$. and the map inherits linearity and boundedness
from $l$. Now, it follows that $\Phi(f) = (x \mapsto f([x]) = (x \mapsto l(x)) = l$, 
and we are done.

\hfill $\qed$

\bigskip


\end{solution}

\newpage

\begin{question}[2]
\hfill
\begin{figure}[h!]
  \centering
    \includegraphics[width=0.7\textwidth]{funcA-h-e1-p2.png}
\end{figure}
\end{question}
\begin{solution} \hfill \\
Fix $x \in E$. For any $y \in G$, and 
$f \in E^*$ such that $||f|| \leq 1$ with $f = 0$ on $G$,
\eQb
|<f,x>| &=& |<f,x-y>| \leq ||f|| ||x-y|| = ||x-y||,
\eQe
so 
\eQb
m(x) &\geq& \sup_{|f|| \leq 1; f = 0 \> \text{on} \> G} |<f,x>|. 
\eQe
Hence, it suffices to show that there exists $f \in E^*$ such that 
$||f|| \leq 1$ and $f = 0$ on $G$ with $|<f,x>| = m(x)$. We can further assume
without loss of generality that $x \in E\setminus G$, because 
if $x \in G$, both sides are trivially $0$. Now, define
a map $g:G \oplus \mathbb{R}x \to 
\mathbb{R}$ by $y + \lambda x \mapsto \lambda m(x)$.
Then, we claim that $g$ is linear. If $z_1 = y_1 + \lambda_1 x$ and
$z_2 = y_2 + \lambda_2 x$, then 
\eQb
g(z_1) + g(z_2) = (\lambda_1 + \lambda_2)x = g(z_1 + z_2).
\eQe
If $z = y + \lambda x$ and $\gamma \in \mathbb{R}$, then
\eQb
g(\gamma z) = g(\gamma y + \gamma \lambda x) = \gamma \lambda m(x) = \gamma g(z). 
\eQe
Note that $g(x) = m(x)$. Now, in view of Hahn-Banach, we certainly need a
Minkowski functional that bounds $g$ on its domain. We show that $m$ in fact
is the Minkowski functional. Firstly, 
\eQb
m(\lambda v) &=& \inf_{y \in G} ||\lambda v - y|| = |\lambda| \inf_{y \in G} 
||v - \dfrac{y}{\lambda}|| = |\lambda| m(v)
\eQe
for any $\lambda \in \mathbb{R}$ 
and $v \in E$, because $y \mapsto \dfrac{y}{\lambda}$ is a
bijection from $G$ to $G$ itself for any $\lambda \in \mathbb{R}$.
Secondly,
\eQb
m(u + v) &=& \inf_{y \in G} ||u + v - y|| \leq \inf_{y \in G} ||u - \dfrac{1}{2}y||
+ \inf_{y \in G} ||v - \dfrac{1}{2} y|| = m(u) + m(v)
\eQe
for any $u,v \in E$.  
where the last equality holds by the same reasoning as above. Finally,
\eQb
g(z) &=& g(y + \lambda x) = \lambda m(x) \leq |\lambda|m(x) = m(y + \lambda x) = m(z) 
\eQe
for any $z = y + \lambda x \in G$. Therefore, by Hahn-Banach, we can extend $g$
to the entire domain $E$ and call it $f$. Since
\eQb
f(z) \leq m(z) \leq ||z||,
\eQe 
for any $z \in E$, it follows that $||f|| \leq 1$, $f = 0$ on $G$ and $|<f,x>| = 0$
as required. \hfill $\qed$

\bigskip

The second part follows similarly. Observe that
\eQb
||g||_{G} = \sup_{||y|| \leq 1; y \in G} |<g,y>| = \sup_{||y|| \leq 1; y \in G}
|<g-h,x>| \leq ||g-h||_{E^*}
\eQe
for any $g \in E^*$ and $h \in G^{\perp}$, so
\eQb
||g||_{G} &\leq& \inf_{h \in G^{\perp}} ||g-h||_{E^*} 
\eQe
for any $g \in E^*$.
Now, $g$ restricted to $G$ is linear and bounded, so by a corollary of 
Hahn-Banach, there exists $f$ such that $||f||_{E^*} = ||g||_{G}$. 
Set $h= g - f$. Since $f = g$ on $G$, $h \in G^{\perp}$, and 
$||g-h||_{E^*} = ||f||_{E^*} = ||g||_{G}$, so we are done. \hfill $\qed$
\end{solution}

\newpage

\begin{question}[3]
\hfill
\begin{figure}[h!]
  \centering
    \includegraphics[width=0.7\textwidth]{funcA-h-e1-p3.png}
\end{figure}
\end{question}
\begin{solution} \hfill \\
\textbf{(i)} We prove the following generalization, known as the Riesz Lemma:
for each $\epsilon > 0$, there exists $x \in E$ such that $||x - y|| 
\geq 1 - \epsilon$, for any $y \in Y$. 

\smallskip

Let $0 < \epsilon < 1$. Let $x \in E\setminus Y$. As $Y$ is closed,
\eQb
d &:=& \text{dist}(x,Y) > 0.
\eQe
Choose $y^*$ in $Y$ such that 
\eQb
d \leq || x - y^* || \leq \dfrac{d}{1-\epsilon}. \>\>\> (1)
\eQe
Set $x^* = \dfrac{x-y^*}{||x-y^*||}$. Clearly, $||x^*|| = 1$, and,
for any $y \in Y$,
\eQb
||x^* - y|| &=& ||\dfrac{x-y^*}{||x-y^*||} - y|| = 
\dfrac{1}{||x-y^*||}||x- (y^* + y||x-y||^*)|| \\
&\geq&  \dfrac{d}{||x-y^*||} \leq 1-\epsilon, 
\eQe 
where the last inequality follows from $(1)$, and we are done. \hfill $\qed$ 

\bigskip

\textbf{(ii)} We proceed to construct a sequence $\{x_n\} \subset B_1$ 
such that there is no convergent subsequence, which shows that $B_1$ is not
compact in strong topology through sequential characterization of compactness
(strong topology is trivially metrizable). 

\smallskip
Choose any $x \in E$ such that $||x|| = 1$ and set $x_1 = x$. 
Then, for any $n$, using $(i)$, 
choose $x_n$ such that 
\eQb
||x_n|| = 1 \>\> \text{and}  \>\> ||x_n - y|| > \dfrac{1}{2}, 
\eQe
for any $y \in \text{span}(x_1,...,x_{n-1})$, where the validity comes from
the fact that any finite dimensional subspace is a proper, closed subspace
of an infinite dimensional space. Then, it is clear that $\{x_n\}$ has no
convergent subsequence, because for any $n \geq 1$, there exists $k, l \geq n$
with $k \neq l$, such that $||x_k - x_l|| > \dfrac{1}{2}$. Since being cauchy
is a necessary condition for being convergent, we are done. 
\hfill $\qed$ 

\end{solution}

\newpage

\begin{question}[4]
\hfill
\begin{figure}[h!]
  \centering
    \includegraphics[width=0.7\textwidth]{funcA-h-e1-p4.png}
\end{figure}
\end{question}
\begin{solution} \hfill \\
\textbf{(i)} 
Let $g$ be a bounded real-valued function and $\lambda > 0$. Then, by
linearity of Lebesgue integration,
\eQb
p(\lambda g) &=& \inf\{ l(h): \lambda g \leq h \in L^{\infty} \} \\
\lambda p(g) &=& \inf\{ l(\lambda h): g \leq h \in L^{\infty} \}.
\eQe
We claim that
\eQb
A := \{ l(h) : \lambda g \leq h \in L^{\infty} \} &=& 
\{ l(\lambda h): g \leq h \in L^{\infty} \} =: B
\eQe
If $\lambda g \leq h \in L^{\infty}$, 
then $ g \leq \dfrac{h}{\lambda} \in L^{\infty}$,
so $l(\lambda \dfrac{h}{\lambda}) = l(\lambda) \in B$. Conversely, if 
$ g \leq h \in L^{\infty}$ then, $\lambda g \leq \lambda h \in L^{\infty}$, 
so $l(\lambda h) \in A$. Hence, $p(\lambda g) = \lambda p(g)$. 

\smallskip

We now show that $p$ is sub-additive. Let $f,g$ be bounded real functions.
Then, for any $h_1, h_2 \in L^{\infty}$ such that $f \leq h_1$ and $g \leq h_2$,
\eQb
f+g &\leq& h_1 + h_2 \in L^{\infty},
\eQe
so, again by linearity of integration, 
\eQb
p(f+g) &\leq& l(h_1 + h_2) = l(h_1) + l(h_2). 
\eQe
Taking infs for $h_1$, then $h_2$, gives
\eQb
p(f+g) &\leq& p(f) + p(g), 
\eQe
as required.

\smallskip
 
For any $f,g$ bounded real-valued functions, 
\eQb
p(f+g) &=& \inf\{ l(f+g) : f+g \leq h , h \in L^{\infty}[0,1]\} \\ 
&=& \inf\{ l(f) + l(g) : f+g \leq h , h \in L^{\infty}[0,1]\}. 
\eQe
Suppose $g \leq 0$. Then, as $0 \in L^{\infty}[0,1]$ and $l(0) = 0$, by definition,
$p(g) \leq 0$. 

\smallskip

We show that $p(f) = l(f)$ if $f \in L^{\infty}[0,1]$. 
For all $h \in L^{\infty}[0,1]$ such that $f \leq h$, then, by monotonicity
of Lebesgue integration, $l(f) \leq l(h)$. Since $f \leq f$ trivially, it follows
that $p(f) = l(f)$. \hfill $\qed$

\bigskip

\textbf{(ii)}
Now, as $l = p$ on $L^{\infty}[0,1]$, by Hahn-Banach, $l$ can be extended to
the entire space of bounded real-valued functions. This shows that 
we can make sense of integration for any bounded functions in a weaker sense,
sacrificing some nice properties, such as countable additivity and so on(probably
if such properties hold, then it will contradict existence of non-measurable sets
by considering appropriate indicators). 
\hfill $\qed$ 
\end{solution}

\newpage

\begin{question}[5]
\hfill
\begin{figure}[h!]
  \centering
    \includegraphics[width=0.7\textwidth]{funcA-h-e1-p5.png}
\end{figure}
\end{question}
\begin{solution} \hfill \\
Without loss of generality, assume $x = 0$ and $x \not \in K$. Let $\{y_n\}$
be the minimizing sequence. As $E$ is Banach, and $K$ is closed, it suffices to 
show that $\{y_n\}$ is cauchy. Let $d = \lim_{n \to \infty}||y_n||$. 
We claim that 
\eQb
||\dfrac{x_n + x_m}{2}|| \to 1 \>\>\> \text{as} \>\>\> n,m \to \infty. 
\eQe
For each $n,m \geq 1$,
\eQb
||\dfrac{x_n + x_m}{2}|| &=& \dfrac{1}{2} 
\dfrac{||y_m||+||y_n||}{||y_n||||y_m||}|| \dfrac{||y_m||y_n}{||y_n|| + ||y_m||}
+ \dfrac{||y_n||y_m}{||y_n|| + ||y_m||}|| \\
&\geq& \dfrac{1}{2} \dfrac{||y_m||+ ||y_n||}{||y_n||||y_m||}d 
\eQe
by convexity of $K$.
The RHS goes to 
$1$ as $n,m \to \infty$ via $d = \lim_{n \to \infty} ||y_n||$ condition, and 
$||\dfrac{x_n + x_m}{2}|| \geq 1$. We have shown the claimed limit. Now,
by uniform convexity, 
$\{x_n\}$ is cauchy. Observe that
\eQb
||y_n - y_m|| &\leq& ||y_n||||x_n - x_m|| + | ||y_n|| - ||y_m|| | ||x_m||
\eQe
for each $n,m \geq 1$. Since the RHS goes to $0$ as $n,m \to \infty$, it implies that
$\{y_n\}$ is cauchy and we are done. \hfill $\qed$

\end{solution}

\newpage

\begin{question}[6]
\hfill
\begin{figure}[h!]
  \centering
    \includegraphics[width=0.7\textwidth]{funcA-h-e1-p6.png}
\end{figure}
\end{question}
\begin{solution} \hfill \\
By ordinary properties of Minkowski functionals, it suffices to show that
for any $x \in E$ and $\lambda \in \mathbb{R}$,
\eQb
p(\lambda x) = |\lambda| p(x) \>\> &\text{and}& \>\>
p(x) = 0 \implies x = 0.
\eQe
We first prove the absolute homogeneity of $p$.
Now, if $\lambda \geq 0$,
then $p(\lambda x) = \lambda p(x)$ by positive homogeneity of Minkowski functionals.
Now, if $\lambda \geq 0$, then, by symmetry, and positive homogeneity again,
we obtain
\eQb
p(\lambda x) = p(-\lambda x) = -\lambda p(x), 
\eQe
which completes the proof of absolute homogeneity. 

\bigskip

Now, observe that, for any $0 <\alpha < \beta$, and $x \in E$, 
\eQb
\alpha^{-1}x \in C &\implies \beta^{-1}x \in C, \>\>\> (**)
\eQe
because by convexity
\eQb
(1- \dfrac{\beta^{-1}}{\alpha^{-1}})0 + \dfrac{\beta^{-1}}{\alpha^{-1}}\alpha^{-1}x
= \beta^{-1}x \in C.
\eQe
Let $x \in E$ such that $p(x) = 0$, Then, by the above discussion,
it follows that
\eQb
\alpha^{-1}x &\in& C \>\>\> (*),
\eQe
for any $\alpha \in (0,\infty)$. Suppose $x \neq 0$, and let $r > 0$ 
large enough that $C \subset B(0,r)$. Then, it follows that, from $(*)$,
$\dfrac{r}{||x||}x \in C$, which contradicts the fact that $C \subset B(0,r)$.
Therefore, $x = 0$ and we are done. \hfill $\qed$

\bigskip

\textbf{(i)} The new norm is equivalent with the original norm. Choose $r > 0$
such that $\overline{B(0,r)} \subset O$. Then,  
\eQb
\dfrac{r}{||x||} &\in& B(0,r) \subset O
\eQe 
and 
\eQb
rp(x) &\leq& ||x||
\eQe
for any $x \in E$.
For the other direction, choose $R > 0$ such that $O \subset B(0,R)$ with
$\partial B(0,R) \cap O = \emptyset$. Then,
\eQb
\dfrac{R}{||x||} x &\not\in& O
\eQe
for any $x \in E$. By the contrapositive of $(**)$, it follows that
\eQb
||x|| &\leq& R p(x)
\eQe
for any $x \in E$. Hence, the claimed norm equivlaence holds.

\textbf{(ii)} Reflexive implies that the space is Banach (so there is no confusion
with definition). 

We prove the following claim: Let $E$ be Banach space that is reflexive. For
any normed linear space, defined by an equivalent norm on $E$, is reflexive.

\bigskip
Let $Q$ be a norm on $E$ such that $E$ is reflexive, and $P$ be a norm
that is equivalent to $Q$. Since the topology induced by $Q$ and $P$ are the same,
we have that $E_{Q}^* = E_{P}^*$, which we denote as $E^*$. Now, we claim that
the dual norms $Q^*$ and $P^*$ on $E^*$, induced by $Q$ and $P$ are again equvialent.
From the norm equivalence and defintion of dual norm, 
\eQb
|f(x)| &\leq& Q^{*}(f) Q(x) 
\eQe
and
\eQb
|f(x)| &\leq& Q^{*}(f) C P(x)
\eQe
for some $C > 0$, and for any $f \in E^*$ and $x \in E$. Therefore, 
\eQb
P^{*}(f) &\leq& CQ^{*}(f) 
\eQe
for any $f \in E^*$. The other direction can be shown similarly, so we have shown
that the induced dual norms are again equvialent.
Now, by the same argument as above, we see that $E_{Q}^{**} = E_{P}^{**}$,
from which it follows that $J_{Q} = J_{P}$. Therefore, by reflexivity assumption,
we have
\eQb
E_{P}^{**} &=& E_{Q}^{**} = J_{Q}(E) = J_{P}(E)
\eQe  
which shows that $(E,P)$ is reflexive as required. Therefore, we have shown that
the new norm is reflexive, whenever the original norm is reflexive. \hfill $\qed$  


 

\end{solution}

\newpage

\begin{question}[7]
\hfill
\begin{figure}[h!]
  \centering
    \includegraphics[width=0.7\textwidth]{funcA-h-e1-p7.png}
\end{figure}
\end{question}
\begin{solution} \hfill \\
Consider $\{f_n\}$ defined by
\eQb
g &\mapsto& \int_{\mathbb{T}} g(t)
e^{-2\pi i nt} dt = \hat{g}(n) \>\>\> (g \in L^{\infty}(\mathbb{T})) 
\eQe
for each $n \geq 1$.
As $L^{\infty}(\mathbb{T}) \subset L^{1}(\mathbb{T})$, 
by Riemann-Lebesgue lemma, for any $g \in L^{\infty}(\mathbb{T})$,
\eQb
f_n(g) = \hat{g}(n) \to 0. 
\eQe
For any $n \geq 1$, $g \in L^{\infty}(\mathbb{T})$ with $||g|| = 1$, 
\eQb
|\hat{g}(n)| &\leq& \int_{\mathbb{T}} |g| \leq ||g||_{\infty} = 1,
\eQe
Now, for any $n \geq 1$, take $g = e^{2\pi int}$ to see
\eQb
|\hat{g}(n)| &=& |\int_{\mathbb{T}} 1dt| = 1,  
\eQe 
so, combined with the above estimate, we have
\eQb
||f_n|| = 1,
\eQe
as required.

\bigskip

It is a well-known fact that $L^{\infty}$ is not separable, given 
that the ambient space is not finite. Instead of appealing to this general result,
we provide a construction of an uncountable family of functions in $L^{\infty}[a,b]$
with $a <b$ 
such that the distance is at least 1 apart between any two elements
to contradict the separability assumption.
Take $[a,b]$ such that $a < b$. Fix $x_0 \in (a,b)$ and consider $
\mathscr{A} = \{1_{B(x_0,r)} \}$
where $0 < r \leq \min(x_0-a,b-x_0)$. Then, the family is uncountable, but for 
any $f,g \in \mathscr{A}$, 
\eQb
||f - g||_{\infty} &=& 1.
\eQe
Now, suppose the space is separable, hence there exists a countable dense subset
$C$. Now, observe that
\eQb
\bigcup_{x \in \mathscr{A}} B(x,\dfrac{1}{2}) \cap C \subset C,
\eQe 
but the left hand side is uncountable, because its an uncountable 
disjoint union of countable sets. Therefore, $L^{\infty}[a,b]$ is not separable. \hfill
$\qed$

 
\end{solution} 


\end{document}
